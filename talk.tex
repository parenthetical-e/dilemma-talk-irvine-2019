\documentclass[10pt]{beamer}
\usepackage{appendixnumberbeamer}
\usepackage{booktabs}
\usepackage[scale=2]{ccicons}
\usepackage{pgfplots}
\usepackage{xspace}
\usepackage{xcolor}
\usepackage{xurl}
\usepackage{caption}

\usetheme[progressbar=frametitle]{metropolis}
\usepgfplotslibrary{dateplot}
\newcommand{\themename}{\textbf{\textsc{metropolis}}\xspace}
\captionsetup[figure]{labelformat=empty}

% ---------------------------------------------------------
\title{A way around the\\ exploration-exploitation dilemma}
% \subtitle{A modern beamer theme}
% \date{\today}
\date{}
\author{Erik J Peterson}
\institute{Research fellow | CoAxLab\\
Carnegie Mellon University\\
\url{robotpuggle.com}}

% ---------------------------------------------------------
\begin{document}
\maketitle

% \begin{frame}{Table of contents.}
%   \setbeamertemplate{section in toc}[sections numbered]
%   \tableofcontents%[hideallsubsections]
% \end{frame}

% ---------------------------------------------------------
% \begin{frame}[fragile]{Astrocytes and cognition!?}
% \begin{itemize}
%     \item Classically astrocytes have been considered neural support cells.
%     \item Growing evidence that astrocytes can directly drive cognition and motor behavior.
    

% \end{itemize}
% \end{frame}
\begin{frame}[fragile]{The dilemma.}
    \begin{quote}
        Should I exploit an available reward, or explore to try a new uncertain action? 
    \end{quote}
\end{frame}

\begin{frame}[fragile]{The dilemma.}
Let's make this more explicitly about reward learning.
\end{frame}

\begin{frame}[fragile]{The dilemma.}
    \begin{quote}
        Should I exploit an available reward, or explore to try find more rewards? 
    \end{quote}
\end{frame}

\begin{frame}[fragile]{Reward is fundamental.}
\begin{itemize}
    \item If you do not eat and drink and so on, you die.
\end{itemize}
\end{frame}

\begin{frame}[fragile]{The dilemma.}
\begin{itemize}
    \item Exploration is the problem.
\end{itemize}
\end{frame}

\begin{frame}[fragile]{The dilemma.}
\begin{itemize}
    \item Exploration is the problem.
    \item There is no optimal solution.
    \item Only average solutions \cite{Thrun1992a,Dayan1996,Findling2018,Gershman2018b}.
\end{itemize}
\end{frame}

\begin{frame}[fragile]{Information is fundamental.}
\begin{itemize}
    % \item If you do not learn about your niche, you die.
    \item Curiosity is not a luxury.
\end{itemize}
\end{frame}

\begin{frame}[fragile]{Information is not a reward.}
\begin{itemize}
    \item Rewards are a conserved resource.
    \item Information is not.
\end{itemize}
\end{frame}

\begin{frame}[fragile]{Information is not a reward.}
\begin{itemize}
    \item Reward value is fixed?
    \item Information value is \textit{never} fixed.
\end{itemize}
\end{frame}

% \begin{figure}
%     \centering
%     \includegraphics[scale=0.2]{images/agn.png} 
%     \caption{
%     \begin{itemize}
%         \item Very little theoretical study of astrocyte computation.
%         \item Focus is on neuron-glia interactions.
%     \end{itemize}}
% \end{figure}

% ---------------------------------------------------------
% \begin{frame}[allowframebreaks]{References}
%   \bibliography{demo}
%   \bibliographystyle{abbrv}
% \end{frame}

% ---------------------------------------------------------
\end{document}
